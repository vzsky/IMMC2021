 As for team sport, the types of matching, such as elimination tournament,  round-robin, or one-player sports, are identical to individual sports. So, as we had discussed in \ref{discuss2}, there are ranking systems that can be adapt for team sports. Our proposed model can be used to determine the rating that each team should get in a tournament. However, finding the GOAT requires additional information. The team performance is dependent on each member, and each member of the team has a different impact on team performance. This requires statistics of action in a match of each individual to determine the impact. The necessary adjustments to the model are to incorporate personal team impact rating, which would come from combining personal statistics that determine helpfulness to the team. Each individual statistic will be weighted according to its importance during the game and combined to get the personal team impact rating. When combined with the team rating, this rating will result in the actual rating that each player should get. After that, we can go back to our previous model and determine the GOAT by combining rating points with the decaying function. For example, if we consider basketball, even though a team can only win by having a higher score than the opponent, many parts of the team must function together to achieve maximum performance. So, it's not enough to only calculate field goal for every player in a team, but many other aspects including block, steal, rebound or even turnover are needed to calculate the impact each player contribute to a team. With the team impact model of each player in a team and a model to give each team a rating on a tournament, it will be possible to determine the greatest player for any team sports. %grammarlied