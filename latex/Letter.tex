\noindent Dear the Director of Top Sport, 

As per your request to find the yearly greatest player and the greatest player of all time, we came up with a mathematical model that helps in determination. Firstly, we argue that just only win-lose results and round reached, which is being used nowadays, is not enough to determine sports players' actual performance. Instead, we came up with the idea of focusing more on the game result. Specifically, close victory and landslide victory also contribute to the performance of the player. 

We estimate the probability of winning the game using the number of games won and total games played. Then, we calculate the probability of winning the match. We believe that the probability of winning the match of the loser is essential to the determination of rating. Since players' performance is most likely distributed normally, we adjust the rating of the loser to a value according to the loser’s probability of winning by using normal distribution curve. Lastly, to prevent over-gaining and over-losing points, we set the maximum difference of rating between the loser and winner of each match. 

However, increasing the player's rating from the beginning of a tournament is not a good idea since the player that lost to the tournament champion in the first round might be better than the player that lost to the champion in the final round. To solve this, we calculate the rating reversely from the champion by setting the champion's rating to 1000, then comparing the rating by the method stated above. Furthermore, The person who did not attend or walkaway or retired in the tournament will not be rated for that specific tournament. 

After finding the greatest player, we determine the GOAT of Single Men’s Tennis using the same rating system. We suggested that each player’s rating should be accumulated over every tournament they played. However, the rating in the past was supposed to have less impact than the latest year. We think that there is some value of rating decay. Formally, at every end of the year, every player's rating will get multiplied by a factor of $\lambda$. We believed that performance in the past plays a more minor role than the present performance but not negligible. For instance, a player that holds runner-up for years could be considered better than a 1 championship. After stacking the rating over the years, the rating value represents combined performance at that period. 

The main advantage of this model is it considers not only the peak performance of players but also the consistency. From our model, even though the cumulative rating was reduced by some proportion over time, it will reflect if the player was good enough to maintain their performance over a while, which means that our result, GOAT of Men’s Tennis, shows the best-skilled player that can also consistently hold their position. 

Which come to our result, the greatest women tennis player that our model has selected is Angelique Kerber, and the men’s GOAT tennis player is Roger Federer. 

We hope that our model will please and fulfill your desire of finding the greatest of each year and the GOAT. We are looking forward to cooperating with you further. 
\\
\\
\noindent Yours faithfully,

\noindent Team Members.  %grammarlied