In this entire world, there are many types of sports and determination of their rankings. Some of them don't wok efficiently especially Tournament system which player meet only few competitors. Elo system, one of the widely used systems, is a good rating protocol for long competition with massive matchmaking. Performance evaluation may not just stop at the win-lose result of the match when it comes to tournament-based matches. In this report, we have illustrated our development of model to find the Greatest Tennis Player in 2018 (Women's Tennis) and GOAT for Men's Tennis using score difference of matches (match toughness) with accumulated rating system over tournament each year.

For model to find the Greatest Women Tennis Player in 2018, firstly we gather full data of match occured during those big 4 tournament. Then we find a point estimator of probability for a player to win each game, after that we calculate the probability for a player to win a match, given that we know the probability that a player will win a game independently. Then we define rating point as a value that can be compare to see how likely a player will win a match against another player. We believed that player performance distributed normally, so we use normal distribution curve to help in defining rating point. Then we assign a rating point for every player in every tournament, accumulate over all four tournaments in 2018, and conclude the result to find the greatest women's tennis player in 2018.

For the result of our model, we obtained that Angelique Kerber is the Greatest Tennis Player in 2018, she won 1 championship and pass through semifinal and quarterfinal on other two tournament. Even though some of result seems surprising at first, we found that the one's who got high rating can beat lots of pro player, even in beginning round. 

There are lots of benefits on this model, it prevent good players from losing point at the very beginning because of the tough match, also, prevent newbies from gaining rating by luck. Performance of each match really does matter. However, some limitations are also occur, but errors normally about the score of losers so it shouldn't have any severe problem on Greatest Player.

Futhermore, we improve this model to find GOAT for Men's Tennis, representative of Tournament match making sports. All Men's Tennis results from 2000-2016 in Grand Slam Tournament are chosen as representative for calculation. All of rating are assign to each individual player in the tournament. Then, we assume that the performance accumulated in player decrease over time. So, the only way that a player can become GOAT is that they must have high rating and also consistent over a period of time.

For the result of GOAT model, we got Federer Roger as the Greatest Of All Time of Men's Tennis. Other explaination and discussion of this result can be seen further in paper. 

Lastly, we discussed a potential way to improve our model to be applicable to team sport by combining individual impact and team performance to get individual performance and rated according to the model.
